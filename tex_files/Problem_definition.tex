\section{Mathematical description of the problem}
In this paper we are interested in electromagnetic problems that involves 
bianisotropic media under time-harmonic excitation, which was studied in 
\cite{kalarickel2020well}.
While the full details of the problem definition and results are available 
in the reference, here we provide a summary of main points in order to 
ease the understanding of the present developments.

The problem is formulated in a domain $\Omega \in \mathbb{R}^3$ 
which has a boundary denoted by $\Gamma$.
The time harmonic sources imply that all the resulting fields 
are in turn time-harmonic and the assumed factor $e^{j\omega t}$ is 
ubiquitous and is suppressed.
The media involved in the problem is linear and time-invariant and 
is considered to be satisfying the following constitutive relations:

    \begin{equation} \label{eq:constitutiveeqn}
      \left\{
        \begin{array}{ll}
          {\bf D} = (1/c_0) \, P \, {\bf E} + L \, {\bf B} 
            & \mbox{ in } \Omega,\\
          {\bf H} = M \, {\bf E} + c_0 \, Q \, {\bf B} 
            & \mbox{ in } \Omega.\\
        \end{array}
      \right.
    \end{equation}

In the above equation, ${\bf E}$, ${\bf B}$, ${\bf D}$ and ${\bf H}$ are 
complex valued functions defined in $\Omega$ and represent, respectively, 
the electric field, magnetic induction, electric displacement, magnetic field 
and $c_0$ is the speed of light in vacuum. 
The space where we will seek ${\bf E}$ and ${\bf H}$ is 
\cite{monkbook} (p. 82; see also p. 69)
%
\begin{equation}
  U=H_{L^2,\Gamma}(\mathrm{curl},\Omega) =
    \{ {\bf v} \in H(\mathrm{curl},\Omega) \; | \;
    {\bf v} \times {\bf n} \in L_t^2(\Gamma) \},
\end{equation}
%
where \cite{monkbook} (p. 48)
%
\begin{equation}
  L_t^2(\Gamma) = \{ {\bf v} \in (L^2(\Gamma))^3 \;|\;
    {\bf v} \cdot {\bf n} = 0 \ \textrm{ almost everywhere on }\Gamma\}.
\end{equation}

Based on Maxwell's equations, boundary conditions and constitutive relations, 
the following variational formulation of the problem can be deduced \cite{bianisotropi_m3as}:
given $\omega>0$, 
  electric and magnetic current densities ${\bf J}_e, {\bf J}_m \in (L^2(\Omega))^3$ 
  and the known term ${\bf f}_R \in L^2_t(\Gamma)$, involved in admittance boundary condition, 
  find ${\bf E} \in U$ such that
  %
  \begin{equation} \label{eq:variationalprob}
    a({\bf E},{\bf v}) = l({\bf v}) \ \ \forall {\bf v} \in U, 
  \end{equation}

\noindent
where
%
\begin{eqnarray} \label{eq:sesquilinear}
  \lefteqn{
    a({\bf u}, {\bf v})
    =
    c_0 
    \big(
      Q \, \mathrm{curl} \, {\bf u}, 
      \mathrm{curl} \, {\bf v}
    \big)_{0,\Omega}
    -
    \frac{\omega^2}{c_0} 
    \big( 
      P \, {\bf u},
      {\bf v}
    \big)_{0,\Omega}
    -
    j \omega
    \big( 
      M \, {\bf u},
      \mathrm{curl} \, {\bf v}
    \big)_{0,\Omega}
  }
  & &
  \nonumber
  \\
  & &
  -
  j \omega
  \big( 
    L \, \mathrm{curl} \, {\bf u},
    {\bf v}
  \big)_{0,\Omega}
  +
  j \omega 
  \big(
    Y \, ({\bf n} \times {\bf u} \times {\bf n}),
    {\bf n} \times {\bf v} \times {\bf n}
  \big)_{0,\Gamma}
  %\ \ {\bf u}, {\bf v} \in U
\end{eqnarray}
%
and
%
\begin{equation} \label{antilinear}
  l({\bf v})
  =
  -
  j \omega 
  \big(
    {\bf J}_e, {\bf v}
  \big)_{0,\Omega}
  -
  c_0
  \big(
    Q \, {\bf J}_m, \mathrm{curl} \, {\bf v}
  \big)_{0,\Omega}
  +
  j \omega 
  \big(
    L \, {\bf J}_m,
    {\bf v}
  \big)_{0,\Omega}
  -
  j \omega 
  \big(
    {\bf f}_R,{\bf n} \times {\bf v} \times {\bf n}
  \big)_{0,\Gamma}.
\end{equation}
%

In \cite{kalarickel2020well} we derived certain sufficient conditions that 
guarantee the well posedness and finite element approximability of the problem.
The developed theory was applied to problems involving rotating axisymmetric objects.
In this paper we apply the theory to a wider range of problems involving bianisotropic media, 
demonstrating the generality of the developments and obtaining interesting new solutions.
In particular we show how the theory can be applied in the presence of metamaterials by 
considering the equivalent media of the type discussed in literature \cite{pendry2016acsphotonics}.

It could be also useful to recall some points on
 how to check for the sufficient conditions that ensure 
the well posedness and finite element approximability of the problem.
In particular it was observed that most of the hypotheses are easily  
verified for important practical problems.
It turns out that the most critical conditions that need to be 
verified are conditions (9) and (10) in \cite{kalarickel2020well}.
These conditions are restated in the following equations. 

\begin{equation}\label{eq:condizione2diteorema2.22delmonk}
  \textrm{for} \ \textrm{every} \ {\bf v} \in U,
  {\bf v} \ne 0, \quad
  \sup_{{\bf u} \in U} \left| a({\bf u}, {\bf v}) \right| > 0,
\end{equation}
%
\begin{equation} \label{eq:infsup}
  \textrm{we} \ \textrm{can} \ \textrm{find} \ \alpha:
  \inf_{ {\bf u} \in U, \ \| {\bf u}\|_U = 1} 
  \sup_{{\bf v} \in U, \ \| {\bf v}\|_U \leq 1} 
    | a({\bf u}, {\bf v}) | \ge \alpha > 0.
\end{equation}
%
As described in \cite{kalarickel2020well}, to satisfy the above conditions,
we need to verify HM9-HM15 in the paper.
Before stating these critical hypotheses a few definitions need to be made.
$\Omega$ is decomposed into $m$ subdomains $\Omega_i$, $i \in I = \{1,2,...m\}$. 
This decomposition can be made such that $I=I_a \cup I_b$, 
where $I_a$ is characterized by subdomains where $L=M=0$. 
Also, an alternative form of constitutive relations given 
in \eqref{eq:constitutiveeqn2} is made use of 
to state some of the hypotheses \cite{noiregolarita}.

\begin{equation} \label{eq:constitutiveeqn2}
  \left\{
    \begin{array}{ll}
      {\bf E} = \kappa \, {\bf D} + \chi \, {\bf B}
        & \mbox{ in } \Omega,\\
      {\bf H} = \gamma \, {\bf D} + \nu \, {\bf B}
        & \mbox{ in } \Omega.\\
    \end{array}
  \right.
\end{equation}

The local continuity of the tensors $P$, $Q$, $L$ and $M$ can be assumed in most 
practical problems and which allows the definiton of the following constants.

\begin{itemize}[leftmargin=*,labelsep=5.8mm] 
  \item 
    $\exists C_L > 0$:
    $| ( L \, \mathrm{curl} \, {\bf u}, {\bf v} )_{0,\Omega} |
    \le 
    C_L \| \mathrm{curl} \, {\bf u} \|_{0,\Omega} \| {\bf v} \|_{0,\Omega}$ 
    for all ${\bf u} \in H(\mathrm{curl}, \Omega)$ and 
    ${\bf v} \in (L^2(\Omega))^3$,
  \item 
    $\exists C_M > 0$:
    $| ( M \, {\bf u}, \mathrm{curl} \, {\bf v} )_{0,\Omega} |
    \le 
    C_M \| {\bf u} \|_{0,\Omega} \| \mathrm{curl} \,{\bf v} \|_{0,\Omega}$ 
    for all ${\bf u} \in (L^2(\Omega))^3$ and 
    ${\bf v} \in H(\mathrm{curl}, \Omega)$.
\end{itemize}
%

Now the important hypotheses are restated here and are renamed as H1-H7.

\begin{description}[leftmargin=32pt,labelsep=7pt]
  \item[H1.]
    $\exists\exists C_{\kappa,d} > 0, \, C_{\nu,d} > 0:
      |determinant\left(\kappa\right)| \ge C_{\kappa,d}, \,
      |determinant\left(\nu\right)| \ge C_{\nu,d}, \, 
      \forall {\bf x} \in \overline{\Omega}_i, \forall i \in I$, 
\end{description}
%
\begin{description}[leftmargin=32pt,labelsep=7pt]
  \item[H2.]
    ${\bf l}_{1,3}^T \, \kappa^{-1} \, {\bf l}_{1,3} \ne 0, \
    {\bf l}_{1,3}^T \, \nu^{-1} \, {\bf l}_{1,3} \ne 0 \
    \forall {\bf l}_{1,3} \in \mathbb{R}^3, {\bf l}_{1,3} \ne 0, \
    \forall {\bf x} \in \overline{\Omega}_i, \, \forall i \in I_a$,
\end{description}
%
\begin{description}[leftmargin=38pt,labelsep=7pt]
  \item[H3.]
    $\exists \exists C_{\kappa,r} > 0, \ C_{\nu,r} > 0: \ 
      |{\bf l}_{1,3,n}^T \, \kappa^{-1} \, {\bf l}_{1,3,n}| \ge C_{\kappa,r}, \
      |{\bf l}_{1,3,n}^T \, \nu^{-1} \, {\bf l}_{1,3,n}| \ge C_{\nu,r}
      \ \forall {\bf l}_{1,3,n} \in \mathbb{R}^3: \| {\bf l}_{1,3,n} \|_2 = 1, \
      \forall {\bf x} \in \overline{\Omega}_i, \, \forall i \in I_b$,
\end{description}
%
\begin{description}[leftmargin=32pt,labelsep=7pt]
  \item[H4.]
    $\exists \exists C_{\kappa,s} > 0, C_{\nu,s} > 0$:
    %
    \begin{equation}
      \big( \sum_{i, j = 1}^3 |\kappa_{ij}| \big) - 
        \min_{i=1,2,3} |\kappa_{ii}| \le C_{\kappa,s}
      \quad \forall {\bf x} \in \overline{\Omega}_k, \, \forall k \in I_b,
    \end{equation}
    %
    \begin{equation}
      \big( \sum_{i, j = 1}^3 |\nu_{ij}| \big) - 
        \min_{i=1,2,3} |\nu_{ii}| \le C_{\nu,s}
      \quad \forall {\bf x} \in \overline{\Omega}_k, \, \forall k \in I_b,
    \end{equation}
    %
    and $\kappa$, $\chi$, $\gamma$ and $\nu$ satisfy
    %
    \begin{equation} \label{condizionesubianisotropixalgogenerale}
      \frac{4 \,
            \Big( \big( \sum_{i, j = 1}^3 |\gamma_{ij}| \big) - 
                  \min_{i=1,2,3} |\gamma_{ii}| \Big)
            \,
            \Big( \big( \sum_{i, j = 1}^3 |\chi_{ij}| \big) - 
                  \min_{i=1,2,3} |\chi_{ii}| \Big)
           }{
        \big( 
          - C_{\kappa,s} + 
          \sqrt{ C_{\kappa,s}^2 + 4 \, C_{\kappa,d} \, C_{\kappa,r}} 
        \big) \, 
        \big( 
          - C_{\nu,s} + 
          \sqrt{ C_{\nu,s}^2 + 4 \, C_{\nu,d} \, C_{\nu,r}} 
        \big)} 
      < 1
    \end{equation}
    %
    $\forall {\bf x} \in \overline{\Omega}_k, \, \forall k \in I_b$.
\end{description}
%

\begin{description}
  \item[H5.]
    We can find $C_{PS} > 0$ such that 
    $|(P {\bf u}, {\bf u})_{0,\Omega}| 
    \ge 
    C_{PS} \| {\bf u} \|_{0,\Omega}^2$ 
    for all ${\bf u} \in (L^2(\Omega))^3$.
\end{description}
%
\begin{description}
  \item[H6.]
    We can find $C_{QS} > 0$ such that 
    $|(Q \mathrm{curl} \, {\bf u}, \mathrm{curl} \, {\bf u})_{0,\Omega}| 
    \ge 
    C_{QS} \| \mathrm{curl} \, {\bf u} \|_{0,\Omega}^2$ 
    for all ${\bf u} \in H(\mathrm{curl}, \Omega)$.
\end{description}
% 

\begin{description}
  \item[H7.]
    $C_{PS}$, $C_{QS}$, $C_L$ and $C_M$ (i.e., all media involved) 
    are such that  $C_{QS} - \frac{C_L C_M}{C_{PS}} > 0$.
\end{description}
%

The section ``some hints to apply the developed theory'' 
of \cite{kalarickel2020well}, provided the guidance to use 
the theory that was developed.
Lemma 1 of the above paper provides a procedure to check 
conditions H5 and H6 and thus estimate the constants 
$C_{PS}$ and $C_{QS}$ which may in turn be used to check H7. 
Let $P$ be decomposed as $P=P_s - jP_{ss}$, and if 
$P_{ss}$ is uniformly positive definite in some region $\Omega_{el}$
and $P_s$ is uniformly positive definite in the complementary region 
$\Omega \setminus \Omega_{el}$, then H5 can be shown to be true 
in the following way.
We can define $C_1 >0$ and $C_2 >0$ as follows.

\begin{equation} \label{eq:c1_constant}
    \int_{\Omega_{el}}{\bf u}^*P_{ss}{\bf u} \geq C_1 \int_{\Omega{el}}|{\bf u}|^2 = C_1 ||{\bf u}||^2_{0, \Omega_{el}} \hbox{   } \forall {\bf u} \in (L^2(\Omega))^3,
\end{equation}

\begin{equation} \label{eq:c2_constant}
    \left|\int_{\Omega \setminus \Omega_{el}} {\bf u}^*P_s{\bf u}\right| \geq C_2||{\bf u}||^2_{0,\Omega \setminus \Omega_{el}}.
\end{equation}

The continuity of $P_s$ in $\Omega_{el}$ allows the definition of $C_3 > 0$ such that 
  \begin{equation} \label{eq:c3_constant}
    \left| \int_{\Omega_{el}} {\bf u}^* P_s {\bf u} \right| \le 
      C_3 \|{\bf u}\|^2_{0,\Omega_{el}}.
  \end{equation}

If $\Omega_{el}=\Omega$, then we can take $C_{PS} = C_1$, if 
$\Omega_{el} = \emptyset$ then $C_{PS} = C_2$ and in other cases

  \begin{equation} \label{eq:cps_metamaterial}
    C_{PS} = 
      \frac{1}{\sqrt{2}} 
      \min 
      \left( 
        \sqrt{(1-\alpha)} C_2, \sqrt{C_1^2+(1-\frac{1}{\alpha}) C_3^2}
      \right),
  \end{equation}
  %
  where $\alpha$ is such that 
  $1 > \alpha > \frac{C_3^2}{C_1^2+C_3^2} > 0$.

Similar considerations can help to estimate $C_{QS}$ as well.

The values obtained above may not the best possible ones than can 
be estimated and the 
condition in H7 can be made less restrictive 
if the estimates of $C_{PS}$ and $C_{QS}$ can be higher.
For example, whenever $P_s$ is uniformly definite 
in $\Omega$, we have $C_4 >0$ such that

\begin{equation} \label{eq:c4_constant}
  \left| \int_{\Omega} {\bf u}^*P_s{\bf u} \right| 
  \geq 
  C_4||{\bf u}||^2_{0,\Omega}.
\end{equation}

In this case, the best value for $C_{PS}$ would be the 
larger among the one in \eqref{eq:cps_metamaterial} and $C_4$.
Based on the definitions, the constants $C_L$ and $C_M$ can
be  estimated and the validity of H7 can be checked.

\begin{equation} \label{equationnumber27}
  C_L = 
    \max_{i \in I_b} 
    \sup_{{\bf x} \in \Omega_i} \sqrt{\lambda_{max}(L^* L)} 
\end{equation}
%
and 
%
\begin{equation} \label{equationnumber28}
  C_M = 
    \max_{i \in I_b} 
    \sup_{{\bf x} \in \Omega_i} \sqrt{\lambda_{max}(M^* M)},
\end{equation}

As for the constants involved in H1-H4,  the following considerations
can be helpful.

\begin{equation} \label{equationnumber29}
  C_{\kappa, d} = 
    \min_{i \in I_b} \inf_{{\bf x} \in \Omega_i} 
    |determinant(\kappa)|,
\end{equation}
%
\begin{equation} \label{equationnumber30}
  C_{\nu, d} = 
    \min_{i \in I_b} \inf_{{\bf x} \in \Omega_i} 
    |determinant(\nu)|,
\end{equation}
%
\begin{equation} \label{equationnumber31}
  C_{\kappa,s} = 
    \max_{i \in I_b} \sup_{{\bf x} \in \Omega_i} 
      \left((\sum_{i,j=1}^3|\kappa_{ij}|)-\mathrm{min}_{i=1,2,3}|\kappa_{ii}| \right),
\end{equation}
%
\begin{equation} \label{equationnumber32}
  C_{\nu,s} = 
    \max_{i \in I_b} \sup_{{\bf x} \in \Omega_i} 
    \left((\sum_{i,j=1}^3|\nu_{ij}|)-\mathrm{min}_{i=1,2,3}|\nu_{ii}| \right).
\end{equation}
%

As for $C_{\kappa, r}$ and $C_{\nu, r}$ the following consideration
might be helpful.
By definition
%
\begin{equation} \label{equationnumber33}
  C_{\kappa, r} = 
    \min_{i \in I_b} \inf_{{\bf x} \in \Omega_i} \quad
    \min_{{\bf l}_{1,3,n} \in \mathbb{R}^3: \| {\bf l}_{1,3,n} \|_2 = 1} 
    \sqrt{\left(
            {\bf l}_{1,3,n}^T \kappa_{is} {\bf l}_{1,3,n}
          \right)^2 
          + 
          \left(
            {\bf l}_{1,3,n}^T \kappa_{iss} {\bf l}_{1,3,n}
          \right)^2},
\end{equation}
%
\begin{equation} \label{equationnumber34}
  C_{\nu, r} = 
    \min_{i \in I_b} \inf_{{\bf x} \in \Omega_i} \quad
    \min_{{\bf l}_{1,3,n} \in \mathbb{R}^3: \| {\bf l}_{1,3,n} \|_2 = 1} 
    \sqrt{\left(
            {\bf l}_{1,3,n}^T \nu_{is} {\bf l}_{1,3,n}
          \right)^2 
          + 
          \left(
            {\bf l}_{1,3,n}^T \nu_{iss} {\bf l}_{1,3,n}
          \right)^2},
\end{equation}
%
where $\kappa_{is}$ and $\kappa_{iss}$ are the symmetric matrices 
obtained by the usual decomposition of $\kappa^{-1}$ and similarly 
$\nu_{is}$ and $\nu_{iss}$ are those corresponding to $\nu^{-1}$. 
If both the symmetric matrices involved in the above expressions are 
semi-definite, then we can deduce the following lower bounds:
%
\begin{equation} \label{equationnumber35}
  C_{\kappa, r} = 
    \min_{i \in I_b} \inf_{{\bf x} \in \Omega_i}
    \sqrt{ \left( \lambda_{min}(\kappa_{is}) \right)^2 
           + 
           \left( \lambda_{min}(\kappa_{iss}) \right)^2},
\end{equation}
%
\begin{equation} \label{equationnumber36}
  C_{\nu, r} = 
    \min_{i \in I_b} \inf_{{\bf x} \in \Omega_i}
    \sqrt{ \left( \lambda_{min}(\nu_{is}) \right)^2 
           + 
           \left( \lambda_{min}(\nu_{iss}) \right)^2}.
\end{equation}
%

If we also define
%
\begin{equation} \label{equationnumber37}
  C_{\chi,s} = 
    \max_{i \in I_b} \sup_{{\bf x} \in \Omega_i} 
      \left((\sum_{i,j=1}^3|\chi_{ij}|)-\mathrm{min}_{i=1,2,3}|\chi_{ii}| \right),
\end{equation}
%
\begin{equation} \label{equationnumber38}
  C_{\gamma,s} = 
    \max_{i \in I_b} \sup_{{\bf x} \in \Omega_i} 
    \left((\sum_{i,j=1}^3|\gamma_{ij}|)-\mathrm{min}_{i=1,2,3}|\gamma_{ii}| \right),
\end{equation}
%
the sufficient condition for the regularity used for proving 
uniqueness can be expressed as
%
\begin{equation} \label{equationnumber39}
  K_{u} = 
  \frac{4 C_{\chi,s} C_{\gamma,s}}{
  \left( -C_{\kappa,s} + \sqrt{C_{\kappa,s}^2+4C_{\kappa,d}C_{\kappa,r}} \right)
  \left( -C_{\nu,s} + \sqrt{C_{\nu,s}^2+4C_{\nu,d}C_{\nu,r}} \right)} < 1.
\end{equation}
%
