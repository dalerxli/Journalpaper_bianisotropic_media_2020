\documentclass[a4paper,twoside]{article}
\usepackage{geometry}
\usepackage{amsfonts}
\usepackage{amsthm}
\usepackage{amsmath}
\usepackage{amssymb}
\usepackage[dvips]{graphicx}
\graphicspath{{../figures/}}

\usepackage{mathtools}
\usepackage{float}
\usepackage[section]{placeins}
\usepackage{caption}
\usepackage{subcaption}
\usepackage{notoccite}
\usepackage{tikz}
\usepackage[titletoc]{appendix}
\usepackage{array}
\usepackage{enumitem}
\usepackage{booktabs,siunitx}
\newcommand{\AxisRotator}[1][rotate=0]{\tikz [x=0.25cm,y=0.60cm,line width=.2ex,-stealth,#1] \draw (0,0) arc (-150:150:1 and 1);}

\newtheorem{problem}{Problem}
\newtheorem{theorem}{Theorem}
\newtheorem{proposition}{Proposition}
\newtheorem{corollary}{Corollary}
\newtheorem{lemma}{Lemma}
\newtheorem{definition}{Definition}
\newtheorem{remark}{Remark}

\newtheorem{hd}{HD}

\title{Analysis of the Finite Element Approximability of Three-Dimensional Time-Harmonic Electromagnetic Problems Involving Bianisotropic Materials and 
Metamaterials}
\author{
 Praveen Kalarickel Ramakrishnan\footnote{Department of Electrical, Electtronic, Telecommunications Engineering and Naval Architecture, University of Genoa, Via Opera Pia 11a, I-16145, Genoa, Italy, email:pravin.nitc@gmail.com} \and 
Mario Rene Clemente Vargas\footnote{Department of Electrical, Electtronic, Telecommunications Engineering and Naval Architecture, University of Genoa, Via Opera Pia 11a, I-16145, Genoa, Italy, email:mario.clemente@unige.it} \and 
Mirco Raffetto\footnote{Department of Electrical, Electtronic, Telecommunications Engineering and Naval Architecture, University of Genoa, Via Opera Pia 11a, I-16145, Genoa, Italy, email:raffetto@dibe.unige.it}
}

\begin{document}

\maketitle
\input{Introduction}
\section{Mathematical description of the problem}
In this paper we are interested in electromagnetic problems that involves 
bianisotropic media under time-harmonic excitation, which was studied in 
\cite{kalarickel2020well}.
While the full details of the problem definition and results are available 
in the reference, here we provide a summary of main points in order to 
ease the understanding of the present developments.

The problem is formulated in a domain $\Omega \in \mathbb{R}^3$ 
which has a boundary denoted by $\Gamma$.
The time harmonic sources imply that all the resulting fields 
are in turn time-harmonic and the assumed factor $e^{j\omega t}$ is 
ubiquitous and is suppressed.
The media involved in the problem is linear and time-invariant and 
is considered to satisfy the following constitutive relations:

    \begin{equation} \label{eq:constitutiveeqn}
      \left\{
        \begin{array}{ll}
          {\bf D} = (1/c_0) \, P \, {\bf E} + L \, {\bf B} 
            & \mbox{ in } \Omega,\\
          {\bf H} = M \, {\bf E} + c_0 \, Q \, {\bf B} 
            & \mbox{ in } \Omega.\\
        \end{array}
      \right.
    \end{equation}

In the above equation, ${\bf E}$, ${\bf B}$, ${\bf D}$ and ${\bf H}$ are 
complex valued functions defined in $\Omega$ and represent, respectively, 
the electric field, magnetic induction, electric displacement, magnetic field 
and $c_0$ is the speed of light in vacuum. 
The space where we will seek ${\bf E}$ and ${\bf H}$ is 
\cite{monkbook} (p. 82; see also p. 69)
%
\begin{equation}
  U=H_{L^2,\Gamma}(\mathrm{curl},\Omega) =
    \{ {\bf v} \in H(\mathrm{curl},\Omega) \; | \;
    {\bf v} \times {\bf n} \in L_t^2(\Gamma) \},
\end{equation}
%
where \cite{monkbook} (p. 48)
%
\begin{equation}
  L_t^2(\Gamma) = \{ {\bf v} \in (L^2(\Gamma))^3 \;|\;
    {\bf v} \cdot {\bf n} = 0 \ \textrm{ almost everywhere on }\Gamma\}.
\end{equation}

Based on Maxwell's equations, boundary conditions and constitutive relations, 
the following variational formulation of the problem can be deduced \cite{bianisotropi_m3as}:
given $\omega>0$, 
  electric and magnetic current densities ${\bf J}_e, {\bf J}_m \in (L^2(\Omega))^3$ 
  and the known term ${\bf f}_R \in L^2_t(\Gamma)$, involved in admittance boundary condition, 
  find ${\bf E} \in U$ such that
  %
  \begin{equation} \label{eq:variationalprob}
    a({\bf E},{\bf v}) = l({\bf v}) \ \ \forall {\bf v} \in U, 
  \end{equation}

\noindent
where
%
\begin{eqnarray} \label{eq:sesquilinear}
  \lefteqn{
    a({\bf u}, {\bf v})
    =
    c_0 
    \big(
      Q \, \mathrm{curl} \, {\bf u}, 
      \mathrm{curl} \, {\bf v}
    \big)_{0,\Omega}
    -
    \frac{\omega^2}{c_0} 
    \big( 
      P \, {\bf u},
      {\bf v}
    \big)_{0,\Omega}
    -
    j \omega
    \big( 
      M \, {\bf u},
      \mathrm{curl} \, {\bf v}
    \big)_{0,\Omega}
  }
  & &
  \nonumber
  \\
  & &
  -
  j \omega
  \big( 
    L \, \mathrm{curl} \, {\bf u},
    {\bf v}
  \big)_{0,\Omega}
  +
  j \omega 
  \big(
    Y \, ({\bf n} \times {\bf u} \times {\bf n}),
    {\bf n} \times {\bf v} \times {\bf n}
  \big)_{0,\Gamma}
  %\ \ {\bf u}, {\bf v} \in U
\end{eqnarray}
%
and
%
\begin{equation} \label{antilinear}
  l({\bf v})
  =
  -
  j \omega 
  \big(
    {\bf J}_e, {\bf v}
  \big)_{0,\Omega}
  -
  c_0
  \big(
    Q \, {\bf J}_m, \mathrm{curl} \, {\bf v}
  \big)_{0,\Omega}
  +
  j \omega 
  \big(
    L \, {\bf J}_m,
    {\bf v}
  \big)_{0,\Omega}
  -
  j \omega 
  \big(
    {\bf f}_R,{\bf n} \times {\bf v} \times {\bf n}
  \big)_{0,\Gamma}.
\end{equation}
%

In \cite{kalarickel2020well} we derived certain sufficient conditions that 
guarantee the well posedness and finite element approximability of the problem.
The developed theory was applied to problems involving rotating axisymmetric objects.
In this paper we apply the theory to a wider range of problems involving bianisotropic media, 
demonstrating the generality of the developments and obtaining interesting new solutions.
In particular we show how the theory can be applied in the presence of metamaterials by 
considering the equivalent media of the type discussed in literature 
\cite{pendry2016acsphotonics}, \cite{wujaggard}, \cite{alottocodecasa}.

It could be also useful to recall some points on
 how to check for the sufficient conditions that ensure 
the well posedness and finite element approximability of the problem.
In particular it was observed that most of the hypotheses are easily  
verified for important practical problems.
It turns out that the most critical conditions that need to be 
verified are conditions (9) and (10) in \cite{kalarickel2020well}.
These conditions are restated in the following equations. 

\begin{equation}\label{eq:condizione2diteorema2.22delmonk}
  \textrm{for} \ \textrm{every} \ {\bf v} \in U,
  {\bf v} \ne 0, \quad
  \sup_{{\bf u} \in U} \left| a({\bf u}, {\bf v}) \right| > 0,
\end{equation}
%
\begin{equation} \label{eq:infsup}
  \textrm{we} \ \textrm{can} \ \textrm{find} \ \alpha:
  \inf_{ {\bf u} \in U, \ \| {\bf u}\|_U = 1} 
  \sup_{{\bf v} \in U, \ \| {\bf v}\|_U \leq 1} 
    | a({\bf u}, {\bf v}) | \ge \alpha > 0.
\end{equation}
%
As described in \cite{kalarickel2020well}, to satisfy the above conditions,
we need to verify HM9-HM15 in the paper.
Before stating these critical hypotheses a few definitions need to be made.
$\Omega$ is decomposed into $m$ subdomains $\Omega_i$, $i \in I = \{1,2,...m\}$. 
This decomposition can be made such that $I=I_a \cup I_b$, 
where $I_a$ is characterized by subdomains where $L=M=0$. 
Also, an alternative form of constitutive relations given 
in \eqref{eq:constitutiveeqn2} is made use of 
to state some of the hypotheses \cite{noiregolarita}.

\begin{equation} \label{eq:constitutiveeqn2}
  \left\{
    \begin{array}{ll}
      {\bf E} = \kappa \, {\bf D} + \chi \, {\bf B}
        & \mbox{ in } \Omega,\\
      {\bf H} = \gamma \, {\bf D} + \nu \, {\bf B}
        & \mbox{ in } \Omega.\\
    \end{array}
  \right.
\end{equation}

The local continuity of the tensors $P$, $Q$, $L$ and $M$ can be assumed in most 
practical problems and which allows the definiton of the following constants.

\begin{itemize}[leftmargin=*,labelsep=5.8mm] 
  \item 
    $\exists C_L > 0$:
    $| ( L \, \mathrm{curl} \, {\bf u}, {\bf v} )_{0,\Omega} |
    \le 
    C_L \| \mathrm{curl} \, {\bf u} \|_{0,\Omega} \| {\bf v} \|_{0,\Omega}$ 
    for all ${\bf u} \in H(\mathrm{curl}, \Omega)$ and 
    ${\bf v} \in (L^2(\Omega))^3$,
  \item 
    $\exists C_M > 0$:
    $| ( M \, {\bf u}, \mathrm{curl} \, {\bf v} )_{0,\Omega} |
    \le 
    C_M \| {\bf u} \|_{0,\Omega} \| \mathrm{curl} \,{\bf v} \|_{0,\Omega}$ 
    for all ${\bf u} \in (L^2(\Omega))^3$ and 
    ${\bf v} \in H(\mathrm{curl}, \Omega)$.
\end{itemize}
%

Now the important hypotheses are restated here and are renamed as H1-H7.

\begin{description}[leftmargin=32pt,labelsep=7pt]
  \item[H1.]
    $\exists\exists C_{\kappa,d} > 0, \, C_{\nu,d} > 0:
      |determinant\left(\kappa\right)| \ge C_{\kappa,d}, \,
      |determinant\left(\nu\right)| \ge C_{\nu,d}, \, 
      \forall {\bf x} \in \overline{\Omega}_i, \forall i \in I$, 
\end{description}
%
\begin{description}[leftmargin=32pt,labelsep=7pt]
  \item[H2.]
    ${\bf l}_{1,3}^T \, \kappa^{-1} \, {\bf l}_{1,3} \ne 0, \
    {\bf l}_{1,3}^T \, \nu^{-1} \, {\bf l}_{1,3} \ne 0 \
    \forall {\bf l}_{1,3} \in \mathbb{R}^3, {\bf l}_{1,3} \ne 0, \
    \forall {\bf x} \in \overline{\Omega}_i, \, \forall i \in I_a$,
\end{description}
%
\begin{description}[leftmargin=38pt,labelsep=7pt]
  \item[H3.]
    $\exists \exists C_{\kappa,r} > 0, \ C_{\nu,r} > 0: \ 
      |{\bf l}_{1,3,n}^T \, \kappa^{-1} \, {\bf l}_{1,3,n}| \ge C_{\kappa,r}, \
      |{\bf l}_{1,3,n}^T \, \nu^{-1} \, {\bf l}_{1,3,n}| \ge C_{\nu,r}
      \ \forall {\bf l}_{1,3,n} \in \mathbb{R}^3: \| {\bf l}_{1,3,n} \|_2 = 1, \
      \forall {\bf x} \in \overline{\Omega}_i, \, \forall i \in I_b$,
\end{description}
%
\begin{description}[leftmargin=32pt,labelsep=7pt]
  \item[H4.]
    $\exists \exists C_{\kappa,s} > 0, C_{\nu,s} > 0$:
    %
    \begin{equation}
      \big( \sum_{i, j = 1}^3 |\kappa_{ij}| \big) - 
        \min_{i=1,2,3} |\kappa_{ii}| \le C_{\kappa,s}
      \quad \forall {\bf x} \in \overline{\Omega}_k, \, \forall k \in I_b,
    \end{equation}
    %
    \begin{equation}
      \big( \sum_{i, j = 1}^3 |\nu_{ij}| \big) - 
        \min_{i=1,2,3} |\nu_{ii}| \le C_{\nu,s}
      \quad \forall {\bf x} \in \overline{\Omega}_k, \, \forall k \in I_b,
    \end{equation}
    %
    and $\kappa$, $\chi$, $\gamma$ and $\nu$ satisfy
    %
    \begin{equation} \label{condizionesubianisotropixalgogenerale}
      \frac{4 \,
            \Big( \big( \sum_{i, j = 1}^3 |\gamma_{ij}| \big) - 
                  \min_{i=1,2,3} |\gamma_{ii}| \Big)
            \,
            \Big( \big( \sum_{i, j = 1}^3 |\chi_{ij}| \big) - 
                  \min_{i=1,2,3} |\chi_{ii}| \Big)
           }{
        \big( 
          - C_{\kappa,s} + 
          \sqrt{ C_{\kappa,s}^2 + 4 \, C_{\kappa,d} \, C_{\kappa,r}} 
        \big) \, 
        \big( 
          - C_{\nu,s} + 
          \sqrt{ C_{\nu,s}^2 + 4 \, C_{\nu,d} \, C_{\nu,r}} 
        \big)} 
      < 1
    \end{equation}
    %
    $\forall {\bf x} \in \overline{\Omega}_k, \, \forall k \in I_b$.
\end{description}
%

\begin{description}
  \item[H5.]
    We can find $C_{PS} > 0$ such that 
    $|(P {\bf u}, {\bf u})_{0,\Omega}| 
    \ge 
    C_{PS} \| {\bf u} \|_{0,\Omega}^2$ 
    for all ${\bf u} \in (L^2(\Omega))^3$.
\end{description}
%
\begin{description}
  \item[H6.]
    We can find $C_{QS} > 0$ such that 
    $|(Q \mathrm{curl} \, {\bf u}, \mathrm{curl} \, {\bf u})_{0,\Omega}| 
    \ge 
    C_{QS} \| \mathrm{curl} \, {\bf u} \|_{0,\Omega}^2$ 
    for all ${\bf u} \in H(\mathrm{curl}, \Omega)$.
\end{description}
% 

\begin{description}
  \item[H7.]
    $C_{PS}$, $C_{QS}$, $C_L$ and $C_M$ (i.e., all media involved) 
    are such that  $C_{QS} - \frac{C_L C_M}{C_{PS}} > 0$.
\end{description}
%

The section ``some hints to apply the developed theory'' 
of \cite{kalarickel2020well}, provided the guidance to use 
the theory that was developed.
Lemma 1 of the above paper provides a procedure to check 
conditions H5 and H6 and thus estimate the constants 
$C_{PS}$ and $C_{QS}$ which may in turn be used to check H7. 
Let $P$ be decomposed as $P=P_s - jP_{ss}$, and if 
$P_{ss}$ is uniformly positive definite in some region $\Omega_{el}$
and $P_s$ is uniformly positive definite in the complementary region 
$\Omega \setminus \Omega_{el}$, then H5 can be shown to be true 
in the following way.
We can define $C_1 >0$ and $C_2 >0$ as follows.

\begin{equation} \label{eq:c1_constant}
    \int_{\Omega_{el}}{\bf u}^*P_{ss}{\bf u} \geq C_1 \int_{\Omega{el}}|{\bf u}|^2 = C_1 ||{\bf u}||^2_{0, \Omega_{el}} \hbox{   } \forall {\bf u} \in (L^2(\Omega))^3,
\end{equation}

\begin{equation} \label{eq:c2_constant}
    \left|\int_{\Omega \setminus \Omega_{el}} {\bf u}^*P_s{\bf u}\right| \geq C_2||{\bf u}||^2_{0,\Omega \setminus \Omega_{el}}.
\end{equation}

The continuity of $P_s$ in $\Omega_{el}$ allows the definition of $C_3 > 0$ such that 
  \begin{equation} \label{eq:c3_constant}
    \left| \int_{\Omega_{el}} {\bf u}^* P_s {\bf u} \right| \le 
      C_3 \|{\bf u}\|^2_{0,\Omega_{el}}.
  \end{equation}

If $\Omega_{el}=\Omega$, then we can take $C_{PS} = C_1$, if 
$\Omega_{el} = \emptyset$ then $C_{PS} = C_2$ and in other cases

  \begin{equation} \label{eq:cps_metamaterial}
    C_{PS} = 
      \frac{1}{\sqrt{2}} 
      \min 
      \left( 
        \sqrt{(1-\alpha)} C_2, \sqrt{C_1^2+(1-\frac{1}{\alpha}) C_3^2}
      \right),
  \end{equation}
  %
  where $\alpha$ is such that 
  $1 > \alpha > \frac{C_3^2}{C_1^2+C_3^2} > 0$.

Similar considerations can help to estimate $C_{QS}$ as well.

The values obtained above may not the best possible ones than can 
be estimated and the 
condition in H7 can be made less restrictive 
if the estimates of $C_{PS}$ and $C_{QS}$ can be higher.
For example, whenever $P_s$ is uniformly definite 
in $\Omega$, we have $C_4 >0$ such that

\begin{equation} \label{eq:c4_constant}
  \left| \int_{\Omega} {\bf u}^*P_s{\bf u} \right| 
  \geq 
  C_4||{\bf u}||^2_{0,\Omega}.
\end{equation}

In this case, the best value for $C_{PS}$ would be the 
larger among the one in \eqref{eq:cps_metamaterial} and $C_4$.
Based on the definitions, the constants $C_L$ and $C_M$ can
be  estimated and the validity of H7 can be checked.

\begin{equation} \label{equation_for_CL}
  C_L = 
    \max_{i \in I_b} 
    \sup_{{\bf x} \in \Omega_i} \sqrt{\lambda_{max}(L^* L)} 
\end{equation}
%
and 
%
\begin{equation} \label{equation_for_CM}
  C_M = 
    \max_{i \in I_b} 
    \sup_{{\bf x} \in \Omega_i} \sqrt{\lambda_{max}(M^* M)},
\end{equation}

As for the constants involved in H1-H4,  the following considerations
can be helpful.

\begin{equation} \label{equation_for_C_kd}
  C_{\kappa, d} = 
    \min_{i \in I_b} \inf_{{\bf x} \in \Omega_i} 
    |determinant(\kappa)|,
\end{equation}
%
\begin{equation} \label{equation_for_C_nud}
  C_{\nu, d} = 
    \min_{i \in I_b} \inf_{{\bf x} \in \Omega_i} 
    |determinant(\nu)|,
\end{equation}
%
\begin{equation} \label{equation_for_C_ks}
  C_{\kappa,s} = 
    \max_{i \in I_b} \sup_{{\bf x} \in \Omega_i} 
      \left((\sum_{i,j=1}^3|\kappa_{ij}|)-\mathrm{min}_{i=1,2,3}|\kappa_{ii}| \right),
\end{equation}
%
\begin{equation} \label{equation_for_C_nus}
  C_{\nu,s} = 
    \max_{i \in I_b} \sup_{{\bf x} \in \Omega_i} 
    \left((\sum_{i,j=1}^3|\nu_{ij}|)-\mathrm{min}_{i=1,2,3}|\nu_{ii}| \right).
\end{equation}
%

As for $C_{\kappa, r}$ and $C_{\nu, r}$ the following consideration
might be helpful.
By definition
%
\begin{equation} \label{equation_for_C_kr_first}
  C_{\kappa, r} = 
    \min_{i \in I_b} \inf_{{\bf x} \in \Omega_i} \quad
    \min_{{\bf l}_{1,3,n} \in \mathbb{R}^3: \| {\bf l}_{1,3,n} \|_2 = 1} 
    \sqrt{\left(
            {\bf l}_{1,3,n}^T \kappa_{is} {\bf l}_{1,3,n}
          \right)^2 
          + 
          \left(
            {\bf l}_{1,3,n}^T \kappa_{iss} {\bf l}_{1,3,n}
          \right)^2},
\end{equation}
%
\begin{equation} \label{equation_for_C_nur_first}
  C_{\nu, r} = 
    \min_{i \in I_b} \inf_{{\bf x} \in \Omega_i} \quad
    \min_{{\bf l}_{1,3,n} \in \mathbb{R}^3: \| {\bf l}_{1,3,n} \|_2 = 1} 
    \sqrt{\left(
            {\bf l}_{1,3,n}^T \nu_{is} {\bf l}_{1,3,n}
          \right)^2 
          + 
          \left(
            {\bf l}_{1,3,n}^T \nu_{iss} {\bf l}_{1,3,n}
          \right)^2},
\end{equation}
%
where $\kappa_{is}$ and $\kappa_{iss}$ are the symmetric matrices 
obtained by the usual decomposition of $\kappa^{-1}$ and similarly 
$\nu_{is}$ and $\nu_{iss}$ are those corresponding to $\nu^{-1}$. 
If both the symmetric matrices involved in the above expressions are 
semi-definite, then we can deduce the following lower bounds:
%
\begin{equation} \label{equation_for_C_kr}
  C_{\kappa, r} = 
    \min_{i \in I_b} \inf_{{\bf x} \in \Omega_i}
    \sqrt{ \left( \lambda_{min}(\kappa_{is}) \right)^2 
           + 
           \left( \lambda_{min}(\kappa_{iss}) \right)^2},
\end{equation}
%
\begin{equation} \label{equation_for_C_nur}
  C_{\nu, r} = 
    \min_{i \in I_b} \inf_{{\bf x} \in \Omega_i}
    \sqrt{ \left( \lambda_{min}(\nu_{is}) \right)^2 
           + 
           \left( \lambda_{min}(\nu_{iss}) \right)^2}.
\end{equation}
%

If we also define
%
\begin{equation} \label{equation_for_C_chis}
  C_{\chi,s} = 
    \max_{i \in I_b} \sup_{{\bf x} \in \Omega_i} 
      \left((\sum_{i,j=1}^3|\chi_{ij}|)-\mathrm{min}_{i=1,2,3}|\chi_{ii}| \right),
\end{equation}
%
\begin{equation} \label{equation_for_C_gammas}
  C_{\gamma,s} = 
    \max_{i \in I_b} \sup_{{\bf x} \in \Omega_i} 
    \left((\sum_{i,j=1}^3|\gamma_{ij}|)-\mathrm{min}_{i=1,2,3}|\gamma_{ii}| \right),
\end{equation}
%
the sufficient condition for the regularity used for proving 
uniqueness can be expressed as
%
\begin{equation} \label{equation_for_Ku}
  K_{u} = 
  \frac{4 C_{\chi,s} C_{\gamma,s}}{
  \left( -C_{\kappa,s} + \sqrt{C_{\kappa,s}^2+4C_{\kappa,d}C_{\kappa,r}} \right)
  \left( -C_{\nu,s} + \sqrt{C_{\nu,s}^2+4C_{\nu,d}C_{\nu,r}} \right)} < 1.
\end{equation}
%

\section{Results and discussion}  
In this section we apply the theory developed in \cite{kalarickel2020well} to 
several different class to problems which could not be managed with the 
previous theory like the one in \cite{bianisotropi_m3as}.
The conditions are established on the parameters of such problems, 
under which the well posedness and finite element 
approximability  can be guaranteed.
Under such condition, the numerical solutions for the fields are 
computed for the first time.
The details of our finite element simulator is the same as that 
described in Section 5 of \cite{kalarickel2020well}.
%In particular let us first consider the class of problems involving materials described in Kraft et al. \cite{pendry2016acsphotonics}.

\input{kraft_pendry}
\subsection{Chirowaveguides considered in \cite{wujaggard}}
In \cite{wujaggard}, the authors consider a metallic waveguide 
which is air filled except for an obstacle
characterized by a chiral media  with the following constitutive relations. 

\begin{equation}
\left\{
\begin{array}{ll}
{\bf D} = \varepsilon_0\varepsilon_r I_3 {\bf E} - j \xi_c I_3 {\bf B} \\
{\bf H} = -j\xi_c I_3 {\bf E} + \frac{1}{\mu_0\mu_r} I_3 {\bf B}.
\end{array}
\right.
\end{equation}

Here $\varepsilon_r$, $\mu_r$ and $\xi_c$ are strictly positive real quantities.
Thus from \eqref{eq:constitutiveeqn} we easily can easily identify $P$, $Q$, $L$
and $M$ which are given below.

\begin{equation} \label{constitutive_wu_P}
P = \varepsilon_0\varepsilon_rc_0 I_3 
\end{equation}

\begin{equation} \label{constitutive_wu_Q}
Q = \frac{1}{\mu_0\mu_rc_0} I_3
\end{equation}

\begin{equation} \label{constitutive_wu_LM}
L = M = -j\xi_cI_3
\end{equation}

In \cite{bianisotropi_m3as} it was shown that this media could not be managed by the previous theory
developed there.
However, the generality of the present theory allows us to apply it to obtain 
the conditions for well-posedness and finite element approximability of this kind 
of problem of practical interest.

Here we analyze the validity of the hypotheses by considering $\varepsilon_r > 1$ 
and $\mu_r =1$.
$P_s$ is equal to $\varepsilon_0 \varepsilon_r c_0$ inside material 
and simply $\varepsilon_0 c_0$ outside.
Since $\Omega_{el} = \emptyset$,  $C_{PS}$ is equal to $C_2$ defined in 
\eqref{eq:c2_constant} and has a value $\varepsilon_0 c_0$, 
verifying the hypothesis H5.
The hypothesis H6 is also trivially valid and $C_{QS} = \frac{1}{\mu_0 c_0}$. 
By equations \eqref{equation_for_CL} and \eqref{equation_for_CM}, $C_L = C_M =  |\xi_c|$.
Then the inequality in hypothesis H7 becomes  
$C_{QS} - \frac{C_L C_M}{C_{PS}} = \frac{1}{c_0}(\frac{1}{\mu_0} - \frac{\xi_c^2}{\varepsilon_0}) > 0$ 
which implies

\begin{equation} \label{eq:inf_sup_wu_jaggard}
\xi_c < \sqrt{\frac{\varepsilon_0}{\mu_0}} = 2.654 \, 10^{-3}  \text{ mho }.
\end{equation}

This is not a small value considering the chiral effects reported in \cite{wujaggard}. 
Now we need to verify the hypotheses H1-H4 which need to hold true locally and hence
we have to examine only media inside the obstacle which is bianisotropic. 
For doing this the suitable form 
of constitutive relations is in terms of $\kappa$, $\nu$, $\gamma$ and $\chi$ 
which are given by the following \cite{noiregolarita}. 

\begin{equation}
\kappa = \frac{1}{\varepsilon_0\varepsilon_r}I_3
\end{equation}

\begin{equation}
\nu = \frac{\varepsilon_0\varepsilon_r+\mu_0\xi_c^2}{\mu_0\varepsilon_0\varepsilon_r}I_3
\end{equation}

\begin{equation}
\chi = -\gamma = \frac{j\xi_c}{\varepsilon_0\varepsilon_r}I_3
\end{equation}

$\kappa$ and $\nu$ are multiples of the identity matrix with eigenvalues $\varepsilon_0\varepsilon_r$ 
and $(\frac{\varepsilon_0\varepsilon_r+\mu_0\xi_c^2}{\mu_0\varepsilon_0\varepsilon_r})$ respectively.
The determinants are just the cubes of the eigenvalues and hence according to equations 
\eqref{equation_for_C_kd} and \eqref{equation_for_C_nud} we get the values of $C_{\kappa,d}$ and 
$C_{\nu,d}$

\begin{equation}
C_{\kappa,d} = \left(\frac{1}{\varepsilon_0\varepsilon_r}\right)^3
\end{equation}

\begin{equation}
C_{\nu,d} = \left(\frac{\varepsilon_0\varepsilon_r+\mu_0\xi_c^2}{\mu_0\varepsilon_0\varepsilon_r}\right)^3
\end{equation}

$C_{\kappa,s}$ and $C_{\nu,s}$ by equations \eqref{equation_for_C_ks} and \eqref{equation_for_C_nus} 
are in this case simply twice the eigenvalue of corresponding diagonal matrix.

\begin{equation}
C_{\kappa,s} = \frac{2}{\varepsilon_0\varepsilon_r}
\end{equation}

\begin{equation}
C_{\nu,s} = 2\frac{\varepsilon_0\varepsilon_r+\mu_0\xi_c^2}{\mu_0\varepsilon_0\varepsilon_r}
\end{equation}

The inverse of the matrices are also trivial and equations \eqref{equation_for_C_kr} and \eqref{equation_for_C_nur} simply
evaluate to the reciprocals of eigenvalues of $\kappa$ and $\nu$ respectively giving $C_{\kappa,r}$ and $C_{\nu,r}$. 

\begin{equation}
C_{\kappa,r} = \varepsilon_0\varepsilon_r
\end{equation}

\begin{equation}
C_{\nu,r} = \frac{\mu_0\varepsilon_0\varepsilon_r}{\varepsilon_0\varepsilon_r+\mu_0\xi_c^2}
\end{equation}

From equations \eqref{equation_for_C_chis} and \eqref{equation_for_C_gammas} we get

\begin{equation}
C_{\chi,s} = C_{\gamma,s} = \frac{\xi_c}{\varepsilon_0\varepsilon_r}.
\end{equation}

Having shown that the hypotheses H1-H3 are satisfied, we can use the above 
constants to calculate $K_u$ and then to verify H4.
Figure \ref{fi:wu_jaggard_regularity_factor_vs_xi} shows the dependence of 
$K_u$ on $\xi_c$ for various values of $\varepsilon_r$.
As the value of $\varepsilon_r$ increases, the hypothesis H4 remains valid
for higher and higher value of $\xi_c$.
Figure \ref{fi:critical_value_of_xi_for_uniqueness_wu_jaggard} shows the
plot of the critical value of $\xi_c$ below which H4 is satisfied against 
$\varepsilon_r$.
The limit of $2.654 \, 10^{-3}$ arising from equation \eqref{eq:inf_sup_wu_jaggard}
required for satisfying H7 is also shown in the same figure.
It is seen that for low values of $\varepsilon_r$ the tighter condition arises from
the need to satisfy H4.
For example the limiting value is $5.6 \, 10^{-4}$ for $\varepsilon_r=1$ and increases
with $\varepsilon_r$ and is $1.78 \, 10^{-3}$ for $\varepsilon_r = 10$.
The curve crosses  $2.654 \, 10^{-3}$ mho line at around $\varepsilon_r = 22.3$ and 
above that value equation \eqref{eq:inf_sup_wu_jaggard} imposes the stricter limit. 

\begin{figure}
\includegraphics{regularity_factor_vs_xi_wu_jaggard.eps}
\caption{Plot of $K_u$ versus $\xi_c$ for the bianisotropic medium described in \cite{wujaggard}.
The plots are shown for various values of $\varepsilon_r$. 
The hypothesis H4 is satisfied for $K_u < 1$.}
\label{fi:wu_jaggard_regularity_factor_vs_xi}
\end{figure}

\begin{figure}
\includegraphics{graphical_analysis_related_to_uniqueness_wu_jaggard.eps}
\caption{The value of $\xi_c$ below which the hypothesis H4
is satisfied is plotted against $\varepsilon_r$.
The limit of $2.654 \, 10^{-3}$ arising from equation \eqref{eq:inf_sup_wu_jaggard}
required for satisfying H7 is also shown.}
\label{fi:critical_value_of_xi_for_uniqueness_wu_jaggard}
\end{figure}

Now we consider a specific numerical problem for which the solution can serve as benchmark 
for other approaches and solvers owing to the reliability of the results guaranteed by the theory.
A rectangular waveguide with a discontinuity due to a block of bianisotropic medium is considered 
as shown in Figure \ref{fi:geometry_wujaggard}.
In the simulation the rectangular waveguide is characterized by  $a=23$ mm, $b=10$ mm  and has a length $l=40$ mm.
The obstacle is a parallelepiped with  $c=11$ mm, $d=5$ mm and length $w=10$ mm.
The origin of the axis is at the lower right corner of the near face of the waveguide shown in Figure \ref{fi:geometry_wujaggard}.
The obstacle ranges from $x = 6$ mm to $x=17$ mm along the x axis, 
from $y=0$ to $y=5$ mm along the y axis and from $z=15$ mm to $z=25$ mm along the z axis.
The bianisotropic medium making up the obstacle is characterized by $\varepsilon_r = 5$ 
and $\xi_c = 1.24\, 10^{-3}$.
For this medium $K_u=0.98 < 1 $ and also equation \eqref{eq:inf_sup_wu_jaggard} is satisfied
and hence all the hypotheses required to guarantee the well posedness and convergence of finite element
solution hold true.
The waveguide is excited with $TE_{10}$ mode with amplitude of 1 $V/m$ and frequency of 9 GHz.

\begin{figure}
\includegraphics[scale=1.2]{draw_waveguide_wu_jaggard.eps}
\caption{The geometry of a rectangular waveguide partially filled with chiral media considered in \cite{wujaggard}.}
\label{fi:geometry_wujaggard}
\end{figure}

The details of the Galerkin finite element solver is the same as before.
The tetrahedral meshes are also obtained in a similar way as discussed in the previous subsection
by dividing the domain into small cubes each of which are in turn subdivided into six tetrahedra.
The convergence of the solution is verified by checking the solutions for three different meshes
which are characterized by small cubes of sides 0.5 mm, 0.25 mm and 0.167 mm which are references as,
respectively, ``coarse'', ``fine'' and ``very fine'' meshes.
There are 10824 nodes, 55200 elements and 6200 boundary faces in coarse mesh,
where as the fine mesh has 79947 nodes, 441600 elements and 24800 boundary faces and 
finally the very fine mesh has 262570 nodes, 1490400 elements and 55800 boundary faces.
The solutions obtained were very stable which is illustrated by showing the results obtained 
for the x component of the magnitude of the electric field along the y axis with
these meshes in Figure \ref{fi:wu_jaggard_convergence}.

\begin{figure}
\includegraphics{figure_convergence_wu_jaggard_along_y_mag_ex.eps}
\caption{Convergence of the solution for problem involving medium in \cite{wujaggard}.
The magnitude of the $x$ component of the electric field is plotted 
along a line parallel to $y$ axis for four different meshes.}
\label{fi:wu_jaggard_convergence}
\end{figure}

Figures  \ref{fi:wu_jaggard_xaxis_ex} to \ref{fi:wu_jaggard_xaxis_ez} show the 
results for the magnitudes and phases of the components of the electric field along
the x axis.
The effect on the fields due to bianisotropic effects are not negligible 
for both the magnitude and phase.
We have, for example, a difference of 20 percent of the magnitude of 
incident field for the x component of the electric field along the x axis. 

\begin{figure}
\centering
\begin{subfigure}[b]{0.49\textwidth}
\includegraphics[width=\textwidth]{figure_wu_jaggard_along_x_mag_ex.eps}
\end{subfigure}
%
\begin{subfigure}[b]{0.49\textwidth}
\centering
\includegraphics[width=\textwidth]{figure_wu_jaggard_along_x_phase_ex.eps}
\end{subfigure}
\caption{The magnitude and phase of the $x$ component of electric field along a line parallel to $x$ axis 
and passing though the center of gravity of the domain for problem involving 
medium in \cite{wujaggard}. 
The plot for bianisotropic case  using $\xi_c = 1.24\,10^{-3}$ mho is compared with 
the solution obtained in isotropic case using $\xi_c = 0$.}
\label{fi:wu_jaggard_xaxis_ex}
\end{figure}

\begin{figure}
\centering
\begin{subfigure}[b]{0.49\textwidth}
\includegraphics[width=\textwidth]{figure_wu_jaggard_along_x_mag_ey.eps}
\end{subfigure}
%
\begin{subfigure}[b]{0.49\textwidth}
\centering
\includegraphics[width=\textwidth]{figure_wu_jaggard_along_x_phase_ey.eps}
\end{subfigure}
\caption{The magnitude and phase of the $y$ component of electric field along a line parallel to $x$ axis 
and passing though the center of gravity of the domain for problem involving 
medium in \cite{wujaggard}. 
The plot for bianisotropic case  using $\xi_c = 1.24\,10^{-3}$ mho is compared with 
the solution obtained in isotropic case using $\xi_c = 0$.}
\label{fi:wu_jaggard_xaxis_ey}
\end{figure}

\begin{figure}
\centering
\begin{subfigure}[b]{0.49\textwidth}
\includegraphics[width=\textwidth]{figure_wu_jaggard_along_x_mag_ez.eps}
\end{subfigure}
%
\begin{subfigure}[b]{0.49\textwidth}
\centering
\includegraphics[width=\textwidth]{figure_wu_jaggard_along_x_phase_ez.eps}
\end{subfigure}
\caption{The magnitude and phase of the $z$ component of electric field along a line parallel to $x$ axis 
and passing though the center of gravity of the domain for problem involving 
medium in \cite{wujaggard}. 
The plot for bianisotropic case  using $\xi_c = 1.24\,10^{-3}$ mho is compared with 
the solution obtained in isotropic case using $\xi_c = 0$.}
\label{fi:wu_jaggard_xaxis_ez}
\end{figure}

Similar results are shown in Figures \ref{fi:wu_jaggard_yaxis_ex} to 
\ref{fi:wu_jaggard_zaxis_ez} for the components of fields along y and 
z directions. 
The bianisotropic effect cause a difference of more than 30 percent of the incident 
field  as can be seen from Figure  \ref{fi:wu_jaggard_yaxis_ex}.

\begin{figure}
\centering
\begin{subfigure}[b]{0.49\textwidth}
\includegraphics[width=\textwidth]{figure_wu_jaggard_along_y_mag_ex.eps}
\end{subfigure}
%
\begin{subfigure}[b]{0.49\textwidth}
\centering
\includegraphics[width=\textwidth]{figure_wu_jaggard_along_y_phase_ex.eps}
\end{subfigure}
\caption{The magnitude and phase of the $x$ component of electric field along a line parallel to $y$ axis 
and passing though the center of gravity of the domain for problem involving 
medium in \cite{wujaggard}. 
The plot for bianisotropic case  using $\xi_c = 1.24\,10^{-3}$ mho is compared with 
the solution obtained in isotropic case using $\xi_c = 0$.}
\label{fi:wu_jaggard_yaxis_ex}
\end{figure}

\begin{figure}
\centering
\begin{subfigure}[b]{0.49\textwidth}
\includegraphics[width=\textwidth]{figure_wu_jaggard_along_y_mag_ey.eps}
\end{subfigure}
%
\begin{subfigure}[b]{0.49\textwidth}
\centering
\includegraphics[width=\textwidth]{figure_wu_jaggard_along_y_phase_ey.eps}
\end{subfigure}
\caption{The magnitude and phase of the $y$ component of electric field along a line parallel to $y$ axis 
and passing though the center of gravity of the domain for problem involving 
medium in \cite{wujaggard}. 
The plot for bianisotropic case  using $\xi_c = 1.24\,10^{-3}$ mho is compared with 
the solution obtained in isotropic case using $\xi_c = 0$.}
\label{fi:wu_jaggard_yaxis_ey}
\end{figure}

\begin{figure}
\centering
\begin{subfigure}[b]{0.49\textwidth}
\includegraphics[width=\textwidth]{figure_wu_jaggard_along_y_mag_ez.eps}
\end{subfigure}
%
\begin{subfigure}[b]{0.49\textwidth}
\centering
\includegraphics[width=\textwidth]{figure_wu_jaggard_along_y_phase_ez.eps}
\end{subfigure}
\caption{The magnitude and phase of the $z$ component of electric field along a line parallel to $y$ axis 
and passing though the center of gravity of the domain for problem involving 
medium in \cite{wujaggard}. 
The plot for bianisotropic case  using $\xi_c = 1.24\,10^{-3}$ mho is compared with 
the solution obtained in isotropic case using $\xi_c = 0$.}
\label{fi:wu_jaggard_yaxis_ez}
\end{figure}

\begin{figure}
\centering
\begin{subfigure}[b]{0.49\textwidth}
\includegraphics[width=\textwidth]{figure_wu_jaggard_along_z_mag_ex.eps}
\end{subfigure}
%
\begin{subfigure}[b]{0.49\textwidth}
\centering
\includegraphics[width=\textwidth]{figure_wu_jaggard_along_z_phase_ex.eps}
\end{subfigure}
\caption{The magnitude and phase of the $x$ component of electric field along a line parallel to $z$ axis 
and passing though the center of gravity of the domain for problem involving 
medium in \cite{wujaggard}. 
The plot for bianisotropic case  using $\xi_c = 1.24\,10^{-3}$ mho is compared with 
the solution obtained in isotropic case using $\xi_c = 0$.}
\label{fi:wu_jaggard_zaxis_ex}
\end{figure}

\begin{figure}
\centering
\begin{subfigure}[b]{0.49\textwidth}
\includegraphics[width=\textwidth]{figure_wu_jaggard_along_z_mag_ey.eps}
\end{subfigure}
%
\begin{subfigure}[b]{0.49\textwidth}
\centering
\includegraphics[width=\textwidth]{figure_wu_jaggard_along_z_phase_ey.eps}
\end{subfigure}
\caption{The magnitude and phase of the $y$ component of electric field along a line parallel to $z$ axis 
and passing though the center of gravity of the domain for problem involving 
medium in \cite{wujaggard}. 
The plot for bianisotropic case  using $\xi_c = 1.24\,10^{-3}$ mho is compared with 
the solution obtained in isotropic case using $\xi_c = 0$.}
\label{fi:wu_jaggard_zaxis_ey}
\end{figure}

\begin{figure}
\centering
\begin{subfigure}[b]{0.49\textwidth}
\includegraphics[width=\textwidth]{figure_wu_jaggard_along_z_mag_ez.eps}
\end{subfigure}
%
\begin{subfigure}[b]{0.49\textwidth}
\centering
\includegraphics[width=\textwidth]{figure_wu_jaggard_along_z_phase_ez.eps}
\end{subfigure}
\caption{The magnitude and phase of the $z$ component of electric field along a line parallel to $z$ axis 
and passing though the center of gravity of the domain for problem involving 
medium in \cite{wujaggard}. 
The plot for bianisotropic case  using $\xi_c = 1.24\,10^{-3}$ mho is compared with 
the solution obtained in isotropic case using $\xi_c = 0$.}
\label{fi:wu_jaggard_zaxis_ez}
\end{figure}

As mentioned earlier, together, these results provide a bench mark for other approaches 
to solving such problems, owing to the reliability of the results provided here which
is guaranteed by the recently developed theory.
The previous theory \cite{bianisotropi_m3as} was not able to manage these problems and 
our results are therefore novel.
The significant bianisotropic effects demonstrated in the results show the practical
importance of the theory for such media.

\subsection{Bianisotropic media considered in \cite{alottocodecasa}}
Another instance of a relevant bianisotropic media was considered in 
\cite{alottocodecasa}, where the authors consider a rectangular waveguide, 
half of which is empty and the other half is filled with a lossless bianisotropic material characterized by 

\begin{equation} \label{constitutive_alotto_P}
P = \varepsilon_0c_0 I_3,
\end{equation}

\begin{equation} \label{constitutive_alotto_Q}
Q = \frac{1}{\mu_0c_0} I_3, 
\end{equation}

\begin{equation} \label{constitutive_alotto_LM}
L = M = j\kappa_0 A ,
\end{equation}
where $A$ is the matrix given by

\begin{equation} \label{constitutive_alotto_LM}
A= 
\begin{bmatrix}
1 & 1 & 0 \\
1 & 1 & 0 \\
0 & 0 & 1 
\end{bmatrix},
\end{equation}
where $\kappa_0$ is a positive real number.

The hypotheses H5 and H6 are trivially valid with $C_{PS} = \varepsilon_0c_0$ and
$C_{QS} = \frac{1}{\mu_0c_0}$.
Further, $L^*L = M^*M =  \kappa_0^2A^2$ whose eigenvalues are 0, $\kappa_0^2$ and $4\kappa_0^2$.
Therefore by equations \eqref{equation_for_CL} and \eqref{equation_for_CM} we get $C_L = C_M = 2\kappa_0$.
The condition in hypothesis H7 then becomes 
$C_{QS} - \frac{C_L C_M}{C_{PS}} = \frac{1}{\mu_0c_0} - \frac{4\kappa_0^2}{\varepsilon_0c_0} > 0$, 
which gives the following limit on $\kappa_0$.

\begin{equation} \label{eq:uniqueness_cond_alotto_codecasa}
\kappa_0 < \frac{1}{2}\sqrt{\frac{\varepsilon_0}{\mu_0}} = 1.327 \, 10^{-3}   \text{ mho }
\end{equation}

The hypotheses H1 to H4 can be studied using the alternative form of constitutive relations
which is characterized by the following matrices \cite{noiregolarita}.

\begin{equation}
\kappa =\frac{1}{\varepsilon_0}I_3
\end{equation}

\begin{equation}
\nu =\frac{1}{\mu_0}I_3 + \frac{\kappa_0^2}{\varepsilon_0}A^2
\end{equation}

\begin{equation}
\chi = -\gamma = \frac{-j\kappa_0}{\varepsilon_0}A
\end{equation}

The determinants of $\kappa$ and $\nu$ can be readily calculated and by using equations
\eqref{equation_for_C_kd} and \eqref{equation_for_C_nud} we get $C_{\kappa,d}$ and $C_{\nu,d}$.

\begin{equation}
C_{\kappa,d} = \frac{1}{\varepsilon_0^3}
\end{equation}

\begin{equation}
C_{\nu,d} =  \frac{(\varepsilon_0+\mu_0\kappa_0^2)(\varepsilon_0+4\mu_0\kappa_0^2)}{\mu_0^3\varepsilon_0^2}
\end{equation}

$C_{\kappa,s}$ and $C_{\nu,s}$ can be directly obtained from their definitions.

\begin{equation}
C_{\kappa,s} = \frac{2}{\varepsilon_0}
\end{equation}

\begin{equation}
C_{\nu,s} = \frac{2\varepsilon_0 + 8\mu_0\kappa_0^2}{\mu_0\varepsilon_0}
\end{equation}

By simple application of the definition
\begin{equation}
C_{\kappa,r} = \varepsilon_0 ,
\end{equation}
and by equation \eqref{equation_for_C_nur}  $C_{\nu,r}$ evaluates to the reciprocal of the largest 
eigenvalue of the real matrix $\nu$ which is given by
\begin{equation}
C_{\nu,r} =  \frac{\mu_0\varepsilon_0}{\varepsilon_0 + 4 \mu_0\kappa_0^2}.
\end{equation}

The equations  \eqref{equation_for_C_chis} and \eqref{equation_for_C_gammas} give 

\begin{equation}
C_{\chi,s} = C_{\gamma,s}= = \frac{4\kappa_0}{\varepsilon_0}.
\end{equation}

The existence of the above constants verifies H1 -H3 and we can use them to 
calculate $K_u$ to verify H4.
It can be verified that $K_u$ is less than one when $\kappa_0 \leq 2.72 \, 10^{-4}$ mho,
which is stricter than the limit obtained from equation \eqref{eq:uniqueness_cond_alotto_codecasa}.

Finally let us give some numerical solutions for this problem, which can be used as benchmarks for other 
approaches.
The cross section of the waveguide is such that it is 2 cm along the x axis and  1 cm along the y axis, and 
the length of the waveguide is 2 cm, half of which is filled with the bianisotropic medium  characterized by 
$\kappa_0 = 2.7 \, 10^{-3}$.
The origin is taken on the corner of the open face of the waveguide on the empty side.
$TE_{10}$ mode is excited in the waveguide with source of amplitude 1 $V/m$ and frequency of 12 GHz.

First order edge element based Galerkin finite element method is used to obtain the solution.
The meshing is carried out by dividing the domain into identical cubes each of which are then 
subdivided into six tetrahedra.
Convergence is ensured by evaluating the solutions on three meshes, 
termed as ``coarse'', ``fine'' and ``very fine'' characterized, respectively, by cubes of sides 1 mm, 
0.5 mm and 0.33 mm.
The coarse mesh had 4852 nodes, 24000 elements and 3200 boundary faces.
The fine mesh is composed of 35301 nodes, 192000 elements and 12800 boundary faces.
The very fine mesh has 115351 nodes, 648000 elements and 28800 boundary faces.
Figure \ref{fi:alotto_codecasa_convergence} shows the results obtained on the three 
meshes for the x component of the electric field along the y axis and confirms the 
stability of the solutions.

\begin{figure}
\includegraphics{figure_convergence_alotto_codecasa_along_y_mag_ex.eps}
\caption{Convergence of the solution for problem involving medium in \cite{alottocodecasa}.
The magnitude of the $x$ component of the electric field is plotted
along a line parallel to $y$ axis for four different meshes.}
\label{fi:alotto_codecasa_convergence}
\end{figure}

We provide the magnitudes and phases of the components of the electric field obtained from the 
simulation in Figures \ref{fi:alotto_codecasa_xaxis_ex} to \ref{fi:alotto_codecasa_zaxis_ez}.
It can be observed that the bianisotropic effects are non negligible and the largest effects 
are on the x component of the field.
The bianisotropy causes a difference in magnitude of up to 14 percent of the incident 
field, as can be seen,  for example, from Figure  \ref{fi:alotto_codecasa_xaxis_ex}, 
\ref{fi:alotto_codecasa_yaxis_ex} or  \ref{fi:alotto_codecasa_zaxis_ex}.
Since the theory guarantees the reliability of these results, it can be used as benchmark for 
other solvers and approaches.


\begin{figure}
\centering
\begin{subfigure}[b]{0.49\textwidth}
\includegraphics[width=\textwidth]{figure_alotto_codecasa_along_x_mag_ex.eps}
\end{subfigure}
%
\begin{subfigure}[b]{0.49\textwidth}
\centering
\includegraphics[width=\textwidth]{figure_alotto_codecasa_along_x_phase_ex.eps}
\end{subfigure}
\caption{The magnitude and phase of the $x$ component of electric field along a line parallel to $x$ axis 
and passing though the center of gravity of the domain for problem involving 
medium in \cite{alottocodecasa}. 
The plot for bianisotropic case  using $\kappa_0 = 2.7\,10^{-4}$ mho is compared with 
the solution obtained in isotropic case using $\kappa_0 = 0$.}
\label{fi:alotto_codecasa_xaxis_ex}
\end{figure}

\begin{figure}
\centering
\begin{subfigure}[b]{0.49\textwidth}
\includegraphics[width=\textwidth]{figure_alotto_codecasa_along_x_mag_ey.eps}
\end{subfigure}
%
\begin{subfigure}[b]{0.49\textwidth}
\centering
\includegraphics[width=\textwidth]{figure_alotto_codecasa_along_x_phase_ey.eps}
\end{subfigure}
\caption{The magnitude and phase of the $y$ component of electric field along a line parallel to $x$ axis 
and passing though the center of gravity of the domain for problem involving 
medium in \cite{alottocodecasa}. 
The plot for bianisotropic case  using $\kappa_0 = 2.7\,10^{-4}$ mho is compared with 
the solution obtained in isotropic case using $\kappa_0 = 0$.}
\label{fi:alotto_codecasa_xaxis_ey}
\end{figure}

\begin{figure}
\centering
\begin{subfigure}[b]{0.49\textwidth}
\includegraphics[width=\textwidth]{figure_alotto_codecasa_along_x_mag_ez.eps}
\end{subfigure}
%
\begin{subfigure}[b]{0.49\textwidth}
\centering
\includegraphics[width=\textwidth]{figure_alotto_codecasa_along_x_phase_ez.eps}
\end{subfigure}
\caption{The magnitude and phase of the $z$ component of electric field along a line parallel to $x$ axis 
and passing though the center of gravity of the domain for problem involving 
medium in \cite{alottocodecasa}. 
The plot for bianisotropic case  using $\kappa_0 = 2.7\,10^{-4}$ mho is compared with 
the solution obtained in isotropic case using $\kappa_0 = 0$.}
\label{fi:alotto_codecasa_xaxis_ez}
\end{figure}

\begin{figure}
\centering
\begin{subfigure}[b]{0.49\textwidth}
\includegraphics[width=\textwidth]{figure_alotto_codecasa_along_y_mag_ex.eps}
\end{subfigure}
%
\begin{subfigure}[b]{0.49\textwidth}
\centering
\includegraphics[width=\textwidth]{figure_alotto_codecasa_along_y_phase_ex.eps}
\end{subfigure}
\caption{The magnitude and phase of the $x$ component of electric field along a line parallel to $y$ axis 
and passing though the center of gravity of the domain for problem involving 
medium in \cite{alottocodecasa}. 
The plot for bianisotropic case  using $\kappa_0 = 2.7\,10^{-4}$ mho is compared with 
the solution obtained in isotropic case using $\kappa_0 = 0$.}
\label{fi:alotto_codecasa_yaxis_ex}
\end{figure}

\begin{figure}
\centering
\begin{subfigure}[b]{0.49\textwidth}
\includegraphics[width=\textwidth]{figure_alotto_codecasa_along_y_mag_ey.eps}
\end{subfigure}
%
\begin{subfigure}[b]{0.49\textwidth}
\centering
\includegraphics[width=\textwidth]{figure_alotto_codecasa_along_y_phase_ey.eps}
\end{subfigure}
\caption{The magnitude and phase of the $y$ component of electric field along a line parallel to $y$ axis 
and passing though the center of gravity of the domain for problem involving 
medium in \cite{alottocodecasa}. 
The plot for bianisotropic case  using $\kappa_0 = 2.7\,10^{-4}$ mho is compared with 
the solution obtained in isotropic case using $\kappa_0 = 0$.}
\label{fi:alotto_codecasa_yaxis_ey}
\end{figure}

\begin{figure}
\centering
\begin{subfigure}[b]{0.49\textwidth}
\includegraphics[width=\textwidth]{figure_alotto_codecasa_along_y_mag_ez.eps}
\end{subfigure}
%
\begin{subfigure}[b]{0.49\textwidth}
\centering
\includegraphics[width=\textwidth]{figure_alotto_codecasa_along_y_phase_ez.eps}
\end{subfigure}
\caption{The magnitude and phase of the $z$ component of electric field along a line parallel to $y$ axis 
and passing though the center of gravity of the domain for problem involving 
medium in \cite{alottocodecasa}. 
The plot for bianisotropic case  using $\kappa_0 = 2.7\,10^{-4}$ mho is compared with 
the solution obtained in isotropic case using $\kappa_0 = 0$.}
\label{fi:alotto_codecasa_yaxis_ez}
\end{figure}

\begin{figure}
\centering
\begin{subfigure}[b]{0.49\textwidth}
\includegraphics[width=\textwidth]{figure_alotto_codecasa_along_z_mag_ex.eps}
\end{subfigure}
%
\begin{subfigure}[b]{0.49\textwidth}
\centering
\includegraphics[width=\textwidth]{figure_alotto_codecasa_along_z_phase_ex.eps}
\end{subfigure}
\caption{The magnitude and phase of the $x$ component of electric field along a line parallel to $z$ axis 
and passing though the center of gravity of the domain for problem involving 
medium in \cite{alottocodecasa}. 
The plot for bianisotropic case  using $\kappa_0 = 2.7\,10^{-4}$ mho is compared with 
the solution obtained in isotropic case using $\kappa_0 = 0$.}
\label{fi:alotto_codecasa_zaxis_ex}
\end{figure}

\begin{figure}
\centering
\begin{subfigure}[b]{0.49\textwidth}
\includegraphics[width=\textwidth]{figure_alotto_codecasa_along_z_mag_ey.eps}
\end{subfigure}
%
\begin{subfigure}[b]{0.49\textwidth}
\centering
\includegraphics[width=\textwidth]{figure_alotto_codecasa_along_z_phase_ey.eps}
\end{subfigure}
\caption{The magnitude and phase of the $y$ component of electric field along a line parallel to $z$ axis 
and passing though the center of gravity of the domain for problem involving 
medium in \cite{alottocodecasa}. 
The plot for bianisotropic case  using $\kappa_0 = 2.7\,10^{-4}$ mho is compared with 
the solution obtained in isotropic case using $\kappa_0 = 0$.}
\label{fi:alotto_codecasa_zaxis_ey}
\end{figure}

\begin{figure}
\centering
\begin{subfigure}[b]{0.49\textwidth}
\includegraphics[width=\textwidth]{figure_alotto_codecasa_along_z_mag_ez.eps}
\end{subfigure}
%
\begin{subfigure}[b]{0.49\textwidth}
\centering
\includegraphics[width=\textwidth]{figure_alotto_codecasa_along_z_phase_ez.eps}
\end{subfigure}
\caption{The magnitude and phase of the $z$ component of electric field along a line parallel to $z$ axis 
and passing though the center of gravity of the domain for problem involving 
medium in \cite{alottocodecasa}. 
The plot for bianisotropic case  using $\kappa_0 = 2.7\,10^{-4}$ mho is compared with 
the solution obtained in isotropic case using $\kappa_0 = 0$.}
\label{fi:alotto_codecasa_zaxis_ez}
\end{figure}


\input{Conclusions}

\bibliographystyle{IEEEtran}
\bibliography{myreferences}
\end{document}
